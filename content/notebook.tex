
% Default to the notebook output style

    


% Inherit from the specified cell style.




    
\documentclass[11pt]{article}

    
    
    \usepackage[T1]{fontenc}
    % Nicer default font (+ math font) than Computer Modern for most use cases
    \usepackage{mathpazo}

    % Basic figure setup, for now with no caption control since it's done
    % automatically by Pandoc (which extracts ![](path) syntax from Markdown).
    \usepackage{graphicx}
    % We will generate all images so they have a width \maxwidth. This means
    % that they will get their normal width if they fit onto the page, but
    % are scaled down if they would overflow the margins.
    \makeatletter
    \def\maxwidth{\ifdim\Gin@nat@width>\linewidth\linewidth
    \else\Gin@nat@width\fi}
    \makeatother
    \let\Oldincludegraphics\includegraphics
    % Set max figure width to be 80% of text width, for now hardcoded.
    \renewcommand{\includegraphics}[1]{\Oldincludegraphics[width=.8\maxwidth]{#1}}
    % Ensure that by default, figures have no caption (until we provide a
    % proper Figure object with a Caption API and a way to capture that
    % in the conversion process - todo).
    \usepackage{caption}
    \DeclareCaptionLabelFormat{nolabel}{}
    \captionsetup{labelformat=nolabel}

    \usepackage{adjustbox} % Used to constrain images to a maximum size 
    \usepackage{xcolor} % Allow colors to be defined
    \usepackage{enumerate} % Needed for markdown enumerations to work
    \usepackage{geometry} % Used to adjust the document margins
    \usepackage{amsmath} % Equations
    \usepackage{amssymb} % Equations
    \usepackage{textcomp} % defines textquotesingle
    % Hack from http://tex.stackexchange.com/a/47451/13684:
    \AtBeginDocument{%
        \def\PYZsq{\textquotesingle}% Upright quotes in Pygmentized code
    }
    \usepackage{upquote} % Upright quotes for verbatim code
    \usepackage{eurosym} % defines \euro
    \usepackage[mathletters]{ucs} % Extended unicode (utf-8) support
    \usepackage[utf8x]{inputenc} % Allow utf-8 characters in the tex document
    \usepackage{fancyvrb} % verbatim replacement that allows latex
    \usepackage{grffile} % extends the file name processing of package graphics 
                         % to support a larger range 
    % The hyperref package gives us a pdf with properly built
    % internal navigation ('pdf bookmarks' for the table of contents,
    % internal cross-reference links, web links for URLs, etc.)
    \usepackage{hyperref}
    \usepackage{longtable} % longtable support required by pandoc >1.10
    \usepackage{booktabs}  % table support for pandoc > 1.12.2
    \usepackage[inline]{enumitem} % IRkernel/repr support (it uses the enumerate* environment)
    \usepackage[normalem]{ulem} % ulem is needed to support strikethroughs (\sout)
                                % normalem makes italics be italics, not underlines
    

    
    
    % Colors for the hyperref package
    \definecolor{urlcolor}{rgb}{0,.145,.698}
    \definecolor{linkcolor}{rgb}{.71,0.21,0.01}
    \definecolor{citecolor}{rgb}{.12,.54,.11}

    % ANSI colors
    \definecolor{ansi-black}{HTML}{3E424D}
    \definecolor{ansi-black-intense}{HTML}{282C36}
    \definecolor{ansi-red}{HTML}{E75C58}
    \definecolor{ansi-red-intense}{HTML}{B22B31}
    \definecolor{ansi-green}{HTML}{00A250}
    \definecolor{ansi-green-intense}{HTML}{007427}
    \definecolor{ansi-yellow}{HTML}{DDB62B}
    \definecolor{ansi-yellow-intense}{HTML}{B27D12}
    \definecolor{ansi-blue}{HTML}{208FFB}
    \definecolor{ansi-blue-intense}{HTML}{0065CA}
    \definecolor{ansi-magenta}{HTML}{D160C4}
    \definecolor{ansi-magenta-intense}{HTML}{A03196}
    \definecolor{ansi-cyan}{HTML}{60C6C8}
    \definecolor{ansi-cyan-intense}{HTML}{258F8F}
    \definecolor{ansi-white}{HTML}{C5C1B4}
    \definecolor{ansi-white-intense}{HTML}{A1A6B2}

    % commands and environments needed by pandoc snippets
    % extracted from the output of `pandoc -s`
    \providecommand{\tightlist}{%
      \setlength{\itemsep}{0pt}\setlength{\parskip}{0pt}}
    \DefineVerbatimEnvironment{Highlighting}{Verbatim}{commandchars=\\\{\}}
    % Add ',fontsize=\small' for more characters per line
    \newenvironment{Shaded}{}{}
    \newcommand{\KeywordTok}[1]{\textcolor[rgb]{0.00,0.44,0.13}{\textbf{{#1}}}}
    \newcommand{\DataTypeTok}[1]{\textcolor[rgb]{0.56,0.13,0.00}{{#1}}}
    \newcommand{\DecValTok}[1]{\textcolor[rgb]{0.25,0.63,0.44}{{#1}}}
    \newcommand{\BaseNTok}[1]{\textcolor[rgb]{0.25,0.63,0.44}{{#1}}}
    \newcommand{\FloatTok}[1]{\textcolor[rgb]{0.25,0.63,0.44}{{#1}}}
    \newcommand{\CharTok}[1]{\textcolor[rgb]{0.25,0.44,0.63}{{#1}}}
    \newcommand{\StringTok}[1]{\textcolor[rgb]{0.25,0.44,0.63}{{#1}}}
    \newcommand{\CommentTok}[1]{\textcolor[rgb]{0.38,0.63,0.69}{\textit{{#1}}}}
    \newcommand{\OtherTok}[1]{\textcolor[rgb]{0.00,0.44,0.13}{{#1}}}
    \newcommand{\AlertTok}[1]{\textcolor[rgb]{1.00,0.00,0.00}{\textbf{{#1}}}}
    \newcommand{\FunctionTok}[1]{\textcolor[rgb]{0.02,0.16,0.49}{{#1}}}
    \newcommand{\RegionMarkerTok}[1]{{#1}}
    \newcommand{\ErrorTok}[1]{\textcolor[rgb]{1.00,0.00,0.00}{\textbf{{#1}}}}
    \newcommand{\NormalTok}[1]{{#1}}
    
    % Additional commands for more recent versions of Pandoc
    \newcommand{\ConstantTok}[1]{\textcolor[rgb]{0.53,0.00,0.00}{{#1}}}
    \newcommand{\SpecialCharTok}[1]{\textcolor[rgb]{0.25,0.44,0.63}{{#1}}}
    \newcommand{\VerbatimStringTok}[1]{\textcolor[rgb]{0.25,0.44,0.63}{{#1}}}
    \newcommand{\SpecialStringTok}[1]{\textcolor[rgb]{0.73,0.40,0.53}{{#1}}}
    \newcommand{\ImportTok}[1]{{#1}}
    \newcommand{\DocumentationTok}[1]{\textcolor[rgb]{0.73,0.13,0.13}{\textit{{#1}}}}
    \newcommand{\AnnotationTok}[1]{\textcolor[rgb]{0.38,0.63,0.69}{\textbf{\textit{{#1}}}}}
    \newcommand{\CommentVarTok}[1]{\textcolor[rgb]{0.38,0.63,0.69}{\textbf{\textit{{#1}}}}}
    \newcommand{\VariableTok}[1]{\textcolor[rgb]{0.10,0.09,0.49}{{#1}}}
    \newcommand{\ControlFlowTok}[1]{\textcolor[rgb]{0.00,0.44,0.13}{\textbf{{#1}}}}
    \newcommand{\OperatorTok}[1]{\textcolor[rgb]{0.40,0.40,0.40}{{#1}}}
    \newcommand{\BuiltInTok}[1]{{#1}}
    \newcommand{\ExtensionTok}[1]{{#1}}
    \newcommand{\PreprocessorTok}[1]{\textcolor[rgb]{0.74,0.48,0.00}{{#1}}}
    \newcommand{\AttributeTok}[1]{\textcolor[rgb]{0.49,0.56,0.16}{{#1}}}
    \newcommand{\InformationTok}[1]{\textcolor[rgb]{0.38,0.63,0.69}{\textbf{\textit{{#1}}}}}
    \newcommand{\WarningTok}[1]{\textcolor[rgb]{0.38,0.63,0.69}{\textbf{\textit{{#1}}}}}
    
    
    % Define a nice break command that doesn't care if a line doesn't already
    % exist.
    \def\br{\hspace*{\fill} \\* }
    % Math Jax compatability definitions
    \def\gt{>}
    \def\lt{<}
    % Document parameters
    \title{data\_clustering\_kmeans}
    
    
    

    % Pygments definitions
    
\makeatletter
\def\PY@reset{\let\PY@it=\relax \let\PY@bf=\relax%
    \let\PY@ul=\relax \let\PY@tc=\relax%
    \let\PY@bc=\relax \let\PY@ff=\relax}
\def\PY@tok#1{\csname PY@tok@#1\endcsname}
\def\PY@toks#1+{\ifx\relax#1\empty\else%
    \PY@tok{#1}\expandafter\PY@toks\fi}
\def\PY@do#1{\PY@bc{\PY@tc{\PY@ul{%
    \PY@it{\PY@bf{\PY@ff{#1}}}}}}}
\def\PY#1#2{\PY@reset\PY@toks#1+\relax+\PY@do{#2}}

\expandafter\def\csname PY@tok@w\endcsname{\def\PY@tc##1{\textcolor[rgb]{0.73,0.73,0.73}{##1}}}
\expandafter\def\csname PY@tok@c\endcsname{\let\PY@it=\textit\def\PY@tc##1{\textcolor[rgb]{0.25,0.50,0.50}{##1}}}
\expandafter\def\csname PY@tok@cp\endcsname{\def\PY@tc##1{\textcolor[rgb]{0.74,0.48,0.00}{##1}}}
\expandafter\def\csname PY@tok@k\endcsname{\let\PY@bf=\textbf\def\PY@tc##1{\textcolor[rgb]{0.00,0.50,0.00}{##1}}}
\expandafter\def\csname PY@tok@kp\endcsname{\def\PY@tc##1{\textcolor[rgb]{0.00,0.50,0.00}{##1}}}
\expandafter\def\csname PY@tok@kt\endcsname{\def\PY@tc##1{\textcolor[rgb]{0.69,0.00,0.25}{##1}}}
\expandafter\def\csname PY@tok@o\endcsname{\def\PY@tc##1{\textcolor[rgb]{0.40,0.40,0.40}{##1}}}
\expandafter\def\csname PY@tok@ow\endcsname{\let\PY@bf=\textbf\def\PY@tc##1{\textcolor[rgb]{0.67,0.13,1.00}{##1}}}
\expandafter\def\csname PY@tok@nb\endcsname{\def\PY@tc##1{\textcolor[rgb]{0.00,0.50,0.00}{##1}}}
\expandafter\def\csname PY@tok@nf\endcsname{\def\PY@tc##1{\textcolor[rgb]{0.00,0.00,1.00}{##1}}}
\expandafter\def\csname PY@tok@nc\endcsname{\let\PY@bf=\textbf\def\PY@tc##1{\textcolor[rgb]{0.00,0.00,1.00}{##1}}}
\expandafter\def\csname PY@tok@nn\endcsname{\let\PY@bf=\textbf\def\PY@tc##1{\textcolor[rgb]{0.00,0.00,1.00}{##1}}}
\expandafter\def\csname PY@tok@ne\endcsname{\let\PY@bf=\textbf\def\PY@tc##1{\textcolor[rgb]{0.82,0.25,0.23}{##1}}}
\expandafter\def\csname PY@tok@nv\endcsname{\def\PY@tc##1{\textcolor[rgb]{0.10,0.09,0.49}{##1}}}
\expandafter\def\csname PY@tok@no\endcsname{\def\PY@tc##1{\textcolor[rgb]{0.53,0.00,0.00}{##1}}}
\expandafter\def\csname PY@tok@nl\endcsname{\def\PY@tc##1{\textcolor[rgb]{0.63,0.63,0.00}{##1}}}
\expandafter\def\csname PY@tok@ni\endcsname{\let\PY@bf=\textbf\def\PY@tc##1{\textcolor[rgb]{0.60,0.60,0.60}{##1}}}
\expandafter\def\csname PY@tok@na\endcsname{\def\PY@tc##1{\textcolor[rgb]{0.49,0.56,0.16}{##1}}}
\expandafter\def\csname PY@tok@nt\endcsname{\let\PY@bf=\textbf\def\PY@tc##1{\textcolor[rgb]{0.00,0.50,0.00}{##1}}}
\expandafter\def\csname PY@tok@nd\endcsname{\def\PY@tc##1{\textcolor[rgb]{0.67,0.13,1.00}{##1}}}
\expandafter\def\csname PY@tok@s\endcsname{\def\PY@tc##1{\textcolor[rgb]{0.73,0.13,0.13}{##1}}}
\expandafter\def\csname PY@tok@sd\endcsname{\let\PY@it=\textit\def\PY@tc##1{\textcolor[rgb]{0.73,0.13,0.13}{##1}}}
\expandafter\def\csname PY@tok@si\endcsname{\let\PY@bf=\textbf\def\PY@tc##1{\textcolor[rgb]{0.73,0.40,0.53}{##1}}}
\expandafter\def\csname PY@tok@se\endcsname{\let\PY@bf=\textbf\def\PY@tc##1{\textcolor[rgb]{0.73,0.40,0.13}{##1}}}
\expandafter\def\csname PY@tok@sr\endcsname{\def\PY@tc##1{\textcolor[rgb]{0.73,0.40,0.53}{##1}}}
\expandafter\def\csname PY@tok@ss\endcsname{\def\PY@tc##1{\textcolor[rgb]{0.10,0.09,0.49}{##1}}}
\expandafter\def\csname PY@tok@sx\endcsname{\def\PY@tc##1{\textcolor[rgb]{0.00,0.50,0.00}{##1}}}
\expandafter\def\csname PY@tok@m\endcsname{\def\PY@tc##1{\textcolor[rgb]{0.40,0.40,0.40}{##1}}}
\expandafter\def\csname PY@tok@gh\endcsname{\let\PY@bf=\textbf\def\PY@tc##1{\textcolor[rgb]{0.00,0.00,0.50}{##1}}}
\expandafter\def\csname PY@tok@gu\endcsname{\let\PY@bf=\textbf\def\PY@tc##1{\textcolor[rgb]{0.50,0.00,0.50}{##1}}}
\expandafter\def\csname PY@tok@gd\endcsname{\def\PY@tc##1{\textcolor[rgb]{0.63,0.00,0.00}{##1}}}
\expandafter\def\csname PY@tok@gi\endcsname{\def\PY@tc##1{\textcolor[rgb]{0.00,0.63,0.00}{##1}}}
\expandafter\def\csname PY@tok@gr\endcsname{\def\PY@tc##1{\textcolor[rgb]{1.00,0.00,0.00}{##1}}}
\expandafter\def\csname PY@tok@ge\endcsname{\let\PY@it=\textit}
\expandafter\def\csname PY@tok@gs\endcsname{\let\PY@bf=\textbf}
\expandafter\def\csname PY@tok@gp\endcsname{\let\PY@bf=\textbf\def\PY@tc##1{\textcolor[rgb]{0.00,0.00,0.50}{##1}}}
\expandafter\def\csname PY@tok@go\endcsname{\def\PY@tc##1{\textcolor[rgb]{0.53,0.53,0.53}{##1}}}
\expandafter\def\csname PY@tok@gt\endcsname{\def\PY@tc##1{\textcolor[rgb]{0.00,0.27,0.87}{##1}}}
\expandafter\def\csname PY@tok@err\endcsname{\def\PY@bc##1{\setlength{\fboxsep}{0pt}\fcolorbox[rgb]{1.00,0.00,0.00}{1,1,1}{\strut ##1}}}
\expandafter\def\csname PY@tok@kc\endcsname{\let\PY@bf=\textbf\def\PY@tc##1{\textcolor[rgb]{0.00,0.50,0.00}{##1}}}
\expandafter\def\csname PY@tok@kd\endcsname{\let\PY@bf=\textbf\def\PY@tc##1{\textcolor[rgb]{0.00,0.50,0.00}{##1}}}
\expandafter\def\csname PY@tok@kn\endcsname{\let\PY@bf=\textbf\def\PY@tc##1{\textcolor[rgb]{0.00,0.50,0.00}{##1}}}
\expandafter\def\csname PY@tok@kr\endcsname{\let\PY@bf=\textbf\def\PY@tc##1{\textcolor[rgb]{0.00,0.50,0.00}{##1}}}
\expandafter\def\csname PY@tok@bp\endcsname{\def\PY@tc##1{\textcolor[rgb]{0.00,0.50,0.00}{##1}}}
\expandafter\def\csname PY@tok@fm\endcsname{\def\PY@tc##1{\textcolor[rgb]{0.00,0.00,1.00}{##1}}}
\expandafter\def\csname PY@tok@vc\endcsname{\def\PY@tc##1{\textcolor[rgb]{0.10,0.09,0.49}{##1}}}
\expandafter\def\csname PY@tok@vg\endcsname{\def\PY@tc##1{\textcolor[rgb]{0.10,0.09,0.49}{##1}}}
\expandafter\def\csname PY@tok@vi\endcsname{\def\PY@tc##1{\textcolor[rgb]{0.10,0.09,0.49}{##1}}}
\expandafter\def\csname PY@tok@vm\endcsname{\def\PY@tc##1{\textcolor[rgb]{0.10,0.09,0.49}{##1}}}
\expandafter\def\csname PY@tok@sa\endcsname{\def\PY@tc##1{\textcolor[rgb]{0.73,0.13,0.13}{##1}}}
\expandafter\def\csname PY@tok@sb\endcsname{\def\PY@tc##1{\textcolor[rgb]{0.73,0.13,0.13}{##1}}}
\expandafter\def\csname PY@tok@sc\endcsname{\def\PY@tc##1{\textcolor[rgb]{0.73,0.13,0.13}{##1}}}
\expandafter\def\csname PY@tok@dl\endcsname{\def\PY@tc##1{\textcolor[rgb]{0.73,0.13,0.13}{##1}}}
\expandafter\def\csname PY@tok@s2\endcsname{\def\PY@tc##1{\textcolor[rgb]{0.73,0.13,0.13}{##1}}}
\expandafter\def\csname PY@tok@sh\endcsname{\def\PY@tc##1{\textcolor[rgb]{0.73,0.13,0.13}{##1}}}
\expandafter\def\csname PY@tok@s1\endcsname{\def\PY@tc##1{\textcolor[rgb]{0.73,0.13,0.13}{##1}}}
\expandafter\def\csname PY@tok@mb\endcsname{\def\PY@tc##1{\textcolor[rgb]{0.40,0.40,0.40}{##1}}}
\expandafter\def\csname PY@tok@mf\endcsname{\def\PY@tc##1{\textcolor[rgb]{0.40,0.40,0.40}{##1}}}
\expandafter\def\csname PY@tok@mh\endcsname{\def\PY@tc##1{\textcolor[rgb]{0.40,0.40,0.40}{##1}}}
\expandafter\def\csname PY@tok@mi\endcsname{\def\PY@tc##1{\textcolor[rgb]{0.40,0.40,0.40}{##1}}}
\expandafter\def\csname PY@tok@il\endcsname{\def\PY@tc##1{\textcolor[rgb]{0.40,0.40,0.40}{##1}}}
\expandafter\def\csname PY@tok@mo\endcsname{\def\PY@tc##1{\textcolor[rgb]{0.40,0.40,0.40}{##1}}}
\expandafter\def\csname PY@tok@ch\endcsname{\let\PY@it=\textit\def\PY@tc##1{\textcolor[rgb]{0.25,0.50,0.50}{##1}}}
\expandafter\def\csname PY@tok@cm\endcsname{\let\PY@it=\textit\def\PY@tc##1{\textcolor[rgb]{0.25,0.50,0.50}{##1}}}
\expandafter\def\csname PY@tok@cpf\endcsname{\let\PY@it=\textit\def\PY@tc##1{\textcolor[rgb]{0.25,0.50,0.50}{##1}}}
\expandafter\def\csname PY@tok@c1\endcsname{\let\PY@it=\textit\def\PY@tc##1{\textcolor[rgb]{0.25,0.50,0.50}{##1}}}
\expandafter\def\csname PY@tok@cs\endcsname{\let\PY@it=\textit\def\PY@tc##1{\textcolor[rgb]{0.25,0.50,0.50}{##1}}}

\def\PYZbs{\char`\\}
\def\PYZus{\char`\_}
\def\PYZob{\char`\{}
\def\PYZcb{\char`\}}
\def\PYZca{\char`\^}
\def\PYZam{\char`\&}
\def\PYZlt{\char`\<}
\def\PYZgt{\char`\>}
\def\PYZsh{\char`\#}
\def\PYZpc{\char`\%}
\def\PYZdl{\char`\$}
\def\PYZhy{\char`\-}
\def\PYZsq{\char`\'}
\def\PYZdq{\char`\"}
\def\PYZti{\char`\~}
% for compatibility with earlier versions
\def\PYZat{@}
\def\PYZlb{[}
\def\PYZrb{]}
\makeatother


    % Exact colors from NB
    \definecolor{incolor}{rgb}{0.0, 0.0, 0.5}
    \definecolor{outcolor}{rgb}{0.545, 0.0, 0.0}



    
    % Prevent overflowing lines due to hard-to-break entities
    \sloppy 
    % Setup hyperref package
    \hypersetup{
      breaklinks=true,  % so long urls are correctly broken across lines
      colorlinks=true,
      urlcolor=urlcolor,
      linkcolor=linkcolor,
      citecolor=citecolor,
      }
    % Slightly bigger margins than the latex defaults
    
    \geometry{verbose,tmargin=1in,bmargin=1in,lmargin=1in,rmargin=1in}
    
    

    \begin{document}
    
    
    \maketitle
    
    

    
    \begin{itemize}
\tightlist
\item
  title: Data clustering basics with k-means algorithm
\item
  author: Oscar Vargas Torres
\item
  date: 2018-06-29
\item
  category: Clustering
\item
  tags: k-means
\end{itemize}

    \hypertarget{data-clustering}{%
\section{Data Clustering}\label{data-clustering}}

Data clustering is a process of assigning a set of records into subsets,
called clusters, such that records in the same cluster are similar and
records in different clusters are quite distinct.

A typical clustering process involves the following five steps:

\begin{enumerate}
\def\labelenumi{\arabic{enumi}.}
\tightlist
\item
  pattern representation;
\item
  dissimilarity measure definition;
\item
  clustering;
\item
  data abstraction;
\item
  assesment of output
\end{enumerate}

In this interactive session, we will be reviewing the \emph{k-means}
algorithm.

    \hypertarget{the-k-means-algorithm}{%
\subsection{\texorpdfstring{The \emph{k-means}
Algorithm}{The k-means Algorithm}}\label{the-k-means-algorithm}}

The \emph{k-means} algorithm is the most popular and the simplest
partitional clustering algorithm. It has many variations. This exercise
will review the standard algorithm and several implementations (possibly
for different variations).

\hypertarget{description-of-the-algorithm}{%
\subsubsection{Description of the
Algorithm}\label{description-of-the-algorithm}}

Let
\(X = \left\{ \mathbf{x}_0, \mathbf{x}_1, \cdots, \mathbf{x}_{n - 1} \right\}\)
be a numeric dataset containing \(n\) records and \(k\) be an integer in
\(\{1, 2, \cdots, n\}\). The \emph{k-means} algorithm tries to divide
the dataset into \(k\) clusters \(C_0\), \(C_1\), \(\cdots\), and
\(C_{k - 1}\) by minimizing the following objective function:
\[E = \sum_{i = 0}^{k - 1}\sum_{\mathbf{x} \in C_i} D(\mathbf{x}, \boldsymbol{\mu}_i),\]
where \(D(\cdot, \cdot)\) is a distance measure and
\(\boldsymbol{\mu}_i\) is the mean of cluster \(C_i\), i.e.,
\[\boldsymbol{\mu}_i = \frac{1}{\lvert C_i \rvert} \sum_{\mathbf{x} \in C_i}\mathbf{x}\]

Let \(\gamma_i\) be the cluster membership of record \(\mathbf{x}_i\)
for \(i = 0\), \(1\), \(\cdots\), \(n - 1\). That is, \(\gamma_i = j\)
if \(\mathbf{x}_i\) belongs to cluster \(C_j\). Then the objective
function can be rewritten as:
\[E = \sum_{i = 0}^{n - 1}D(\mathbf{x}_i, \boldsymbol{\mu}_{\gamma_i})\]

To minimize the objective function, the \emph{k-means} algorithm employs
an iterative process. At the beginning, the \emph{k-means} algorithm
selects \(k\) random records from the dataset \(X\) as initial cluster
centers.

Suppose \(\boldsymbol{\mu}_0^{(0)}\), \(\boldsymbol{\mu}_1^{(0)}\),
\(\cdots\), and \(\boldsymbol{\mu}_{k - 1}^{(0)}\) are the initial
cluster centers. Based on these cluster centers, the \emph{k-means}
algorithm updates the cluster memberships \(\gamma_0^{(0)}\),
\(\gamma_1^{(0)}\), \(\cdots\), \(\gamma_{n - 1}^{(0)}\) as follows:
\begin{equation}
\label{membership-update}
\gamma_i^{(0)} = \underset{0 \leq j \leq k - 1}{\operatorname{argmin}} D(\mathbf{x}_i, \boldsymbol{\mu}_j^{(0)})
\end{equation} where \(\operatorname{argmin}\) is the argument that
minimizes the distance. That is, \(\gamma_i^{(0)}\) is set to the index
of the cluster to which \(\mathbf{x}_i\) has the smallest distance.

Based on the cluster memberships \(\gamma_0^{(0)}\), \(\gamma_1^{(0)}\),
\(\cdots\), \(\gamma_{n - 1}^{(0)}\), the \emph{k-means} algorithm
updates the cluster centers as follows: \begin{equation}
\label{center-update}
\boldsymbol{\boldsymbol{\mu}}_j^{(1)} = \frac{1}{\left| \left\{ i \middle| \gamma_i^{(0)} = j\right\} \right|} \sum_{i = 0, \gamma_i^{(0)} = j}\mathbf{x}_i, \qquad\text{$j = 0$, $1$, $\cdots$, $k - 1$}
\end{equation}

Then the \emph{k-means} algorithm repeats updating the cluster
memberships based on equations \ref{membership-update} and
\ref{center-update}.

    \hypertarget{exercises}{%
\subsection{Exercises}\label{exercises}}

\hypertarget{visualization-of-dataset-for-exercises}{%
\subsubsection{Visualization of dataset for
exercises}\label{visualization-of-dataset-for-exercises}}

    \begin{Verbatim}[commandchars=\\\{\}]
{\color{incolor}In [{\color{incolor}1}]:} \PY{o}{\PYZpc{}}\PY{k}{matplotlib} inline
\end{Verbatim}


    \begin{Verbatim}[commandchars=\\\{\}]
{\color{incolor}In [{\color{incolor}2}]:} \PY{k+kn}{import} \PY{n+nn}{pandas} \PY{k}{as} \PY{n+nn}{pd}
        
        \PY{n}{input\PYZus{}file} \PY{o}{=} \PY{l+s+s2}{\PYZdq{}}\PY{l+s+s2}{data/600points.csv}\PY{l+s+s2}{\PYZdq{}}
        
        \PY{c+c1}{\PYZsh{} Input file has labeled records. \PYZsq{}label\PYZsq{} is one of 1, 2 or 3.}
        \PY{n}{labeled\PYZus{}df} \PY{o}{=} \PY{n}{pd}\PY{o}{.}\PY{n}{read\PYZus{}csv}\PY{p}{(}\PY{n}{input\PYZus{}file}\PY{p}{,}
                                 \PY{n}{header}\PY{o}{=}\PY{k+kc}{None}\PY{p}{,}
                                 \PY{n}{names}\PY{o}{=}\PY{p}{[}\PY{l+s+s1}{\PYZsq{}}\PY{l+s+s1}{id}\PY{l+s+s1}{\PYZsq{}}\PY{p}{,} \PY{l+s+s1}{\PYZsq{}}\PY{l+s+s1}{x}\PY{l+s+s1}{\PYZsq{}}\PY{p}{,} \PY{l+s+s1}{\PYZsq{}}\PY{l+s+s1}{y}\PY{l+s+s1}{\PYZsq{}}\PY{p}{,} \PY{l+s+s1}{\PYZsq{}}\PY{l+s+s1}{label}\PY{l+s+s1}{\PYZsq{}}\PY{p}{]}\PY{p}{)}
        \PY{n}{labeled\PYZus{}df}\PY{o}{.}\PY{n}{dtypes}
\end{Verbatim}


\begin{Verbatim}[commandchars=\\\{\}]
{\color{outcolor}Out[{\color{outcolor}2}]:} id         int64
        x        float64
        y        float64
        label      int64
        dtype: object
\end{Verbatim}
            
    \begin{Verbatim}[commandchars=\\\{\}]
{\color{incolor}In [{\color{incolor}3}]:} \PY{c+c1}{\PYZsh{} To visualize \PYZdq{}named\PYZdq{} colors see:}
        \PY{c+c1}{\PYZsh{} https://matplotlib.org/1.4.3/examples/color/named\PYZus{}colors.html}
        \PY{n}{label\PYZus{}to\PYZus{}color} \PY{o}{=} \PY{p}{\PYZob{}}\PY{l+m+mi}{1}\PY{p}{:} \PY{l+s+s1}{\PYZsq{}}\PY{l+s+s1}{red}\PY{l+s+s1}{\PYZsq{}}\PY{p}{,} \PY{l+m+mi}{2}\PY{p}{:} \PY{l+s+s1}{\PYZsq{}}\PY{l+s+s1}{deepskyblue}\PY{l+s+s1}{\PYZsq{}}\PY{p}{,} \PY{l+m+mi}{3}\PY{p}{:} \PY{l+s+s1}{\PYZsq{}}\PY{l+s+s1}{green}\PY{l+s+s1}{\PYZsq{}}\PY{p}{\PYZcb{}}
        \PY{n}{colors} \PY{o}{=} \PY{n}{labeled\PYZus{}df}\PY{o}{.}\PY{n}{loc}\PY{p}{[}\PY{p}{:}\PY{p}{,} \PY{l+s+s1}{\PYZsq{}}\PY{l+s+s1}{label}\PY{l+s+s1}{\PYZsq{}}\PY{p}{]}\PY{o}{.}\PY{n}{map}\PY{p}{(}\PY{n}{label\PYZus{}to\PYZus{}color}\PY{p}{)}
        \PY{n}{labeled\PYZus{}df}\PY{o}{.}\PY{n}{plot}\PY{o}{.}\PY{n}{scatter}\PY{p}{(}\PY{l+s+s1}{\PYZsq{}}\PY{l+s+s1}{x}\PY{l+s+s1}{\PYZsq{}}\PY{p}{,} \PY{l+s+s1}{\PYZsq{}}\PY{l+s+s1}{y}\PY{l+s+s1}{\PYZsq{}}\PY{p}{,} \PY{n}{grid}\PY{o}{=}\PY{k+kc}{True}\PY{p}{,} \PY{n}{c}\PY{o}{=}\PY{n}{colors}\PY{p}{)}\PY{p}{;}
\end{Verbatim}


    \begin{center}
    \adjustimage{max size={0.9\linewidth}{0.9\paperheight}}{output_6_0.png}
    \end{center}
    { \hspace*{\fill} \\}
    
    \begin{Verbatim}[commandchars=\\\{\}]
{\color{incolor}In [{\color{incolor}4}]:} \PY{c+c1}{\PYZsh{} Unlabeled data (prior to \PYZdq{}clustering\PYZdq{}).}
        \PY{n}{df} \PY{o}{=} \PY{n}{labeled\PYZus{}df}\PY{o}{.}\PY{n}{loc}\PY{p}{[}\PY{p}{:}\PY{p}{,}\PY{p}{[}\PY{l+s+s1}{\PYZsq{}}\PY{l+s+s1}{x}\PY{l+s+s1}{\PYZsq{}}\PY{p}{,} \PY{l+s+s1}{\PYZsq{}}\PY{l+s+s1}{y}\PY{l+s+s1}{\PYZsq{}}\PY{p}{]}\PY{p}{]}
        \PY{n}{df}\PY{o}{.}\PY{n}{plot}\PY{o}{.}\PY{n}{scatter}\PY{p}{(}\PY{l+s+s1}{\PYZsq{}}\PY{l+s+s1}{x}\PY{l+s+s1}{\PYZsq{}}\PY{p}{,} \PY{l+s+s1}{\PYZsq{}}\PY{l+s+s1}{y}\PY{l+s+s1}{\PYZsq{}}\PY{p}{,} \PY{n}{grid}\PY{o}{=}\PY{k+kc}{True}\PY{p}{)}\PY{p}{;}
\end{Verbatim}


    \begin{center}
    \adjustimage{max size={0.9\linewidth}{0.9\paperheight}}{output_7_0.png}
    \end{center}
    { \hspace*{\fill} \\}
    
    \hypertarget{to-practice-your-python}{%
\subsubsection{To practice your Python}\label{to-practice-your-python}}

\begin{enumerate}
\def\labelenumi{\arabic{enumi}.}
\item
  From the datasets card in trello, use data.tar.gz to get the file with
  name 600points.csv.
\item
  Use the kmeans implementation from scikit-learn to run the algorithm
  for k = 3 clusters and a max of 100 iterations. You can install that
  using pip (\texttt{\$\ pip\ install\ scikit-learn}) or anaconda,
  (\texttt{\$\ conda\ install\ scikit-learn}).\\
\item
  Use python's \href{https://github.com/docopt/docopt}{docopt} or
  \texttt{argparse} (available in the standard library) to parse the
  following command line options:

\begin{verbatim}
Allowed options:
  --help                produce help message
  --datafile arg        the data file
  --k arg (=3)          number of clusters
  --maxiter arg (=100)  maximum number of iterations
\end{verbatim}
\item
  Implement the algorithm by yourself and compare with the results
  obtained with scikit-learn.
\item
  Use \href{https://numba.pydata.org/}{numba} to try to improve the
  performance of your implementation.
\item
  Use \href{https://matplotlib.org/}{matplotlib} (or another library of
  your preference) to visualize data and your results. Fine tune your
  plot so that the assesment of the result is easier to understand.
\end{enumerate}

\hypertarget{to-practice-with-.net-c-or-f}{%
\subsubsection{To practice with .NET (C\# or
F\#)}\label{to-practice-with-.net-c-or-f}}

\begin{enumerate}
\def\labelenumi{\arabic{enumi}.}
\tightlist
\item
  Use the \emph{k-means} implementation from version 0.2 of
  \href{https://github.com/dotnet/machinelearning}{ML.NET} as a
  reference implementation for comparison (correctness of results,
  performance, etc).
\item
  Use \href{https://github.com/docopt/docopt.net}{.NET's docopt port} to
  parse command line options for your program (see analogous section for
  Python).
\item
  Implement your own version of k-means using your preferred .NET
  language (C\#, F\#, VB.NET).
\item
  Visualize your results with
  \href{https://github.com/fslaborg/XPlot}{XPlot},
  \href{https://github.com/fslaborg/FSharp.Charting}{FSharp.Charting},
  \href{https://github.com/Live-Charts/Live-Charts}{Live-Charts} or any
  other good quality library of your preference.
\end{enumerate}

\hypertarget{to-practice-with-gpus-and-pythonrc}{%
\subsubsection{To practice with GPUs and
Python/R/C}\label{to-practice-with-gpus-and-pythonrc}}

\begin{enumerate}
\def\labelenumi{\arabic{enumi}.}
\tightlist
\item
  Take a look at \href{https://github.com/src-d/kmcuda}{kmcuda}. See
  what the possibilities are for your favorite programming language.
\item
  Visualize your results with the best library for your language of
  choice.
\end{enumerate}

\hypertarget{to-practice-with-gpus-or-multicore-cpus-and-haskell}{%
\subsubsection{To practice with GPUs or Multicore CPUs and
Haskell}\label{to-practice-with-gpus-or-multicore-cpus-and-haskell}}

\begin{enumerate}
\def\labelenumi{\arabic{enumi}.}
\tightlist
\item
  Take a look at \href{http://www.acceleratehs.org/}{Accelerate} and the
  different supported backends (e.g.~Multicore CPUs and GPUs).
\item
  Study and explain the
  \href{https://github.com/AccelerateHS/accelerate-examples/blob/master/examples/kmeans/Kmeans.hs}{\emph{k-means}
  sample}
\item
  Compare the implementations and performance.
\item
  Visualize using
  \href{https://github.com/abarbu/matplotlib-haskell}{\texttt{matplotlib}
  bindings for Haskell}.
\end{enumerate}

\hypertarget{to-practice-with-scala-and-spark}{%
\subsubsection{To practice with Scala and
Spark}\label{to-practice-with-scala-and-spark}}

\begin{enumerate}
\def\labelenumi{\arabic{enumi}.}
\item
  Try to use the existing Scala implementations:

  \begin{itemize}
  \tightlist
  \item
    With
    \href{https://github.com/apache/spark/tree/master/mllib/src/main/scala/org/apache/spark/mllib/clustering}{Spark's
    MLLib}
  \item
    With
    \href{https://github.com/non/spire/blob/master/examples/src/main/scala/spire/example/kmeans.scala}{Spire}
  \end{itemize}
\item
  Study and explain.
\item
  Hard: study if it is possible to improve performance using
  \href{https://github.com/bytedeco/sbt-javacpp}{sbt-javacpp} bindings
  for CUDA and previous implementation code/algorithms.
\item
  Visualize using \href{https://github.com/vegas-viz/Vegas}{Vegas} or
  any other good quality library of your preference.
\end{enumerate}


    % Add a bibliography block to the postdoc
    
    
    
    \end{document}
